\section{Discussion}
%\textsc{infernal} 1.0's heuristic acceleration techniques make
%important RNA sequence analysis applications feasible without
%sacrificing the sensitivity of unconstrained CM methods. 

\textsc{infernal} is a sensitive tool for RNA sequence analysis.
Version 1.0's heuristic acceleration techniques make some important
applications possible on a single desktop computer in less than
an hour, such as searching a prokaryotic genome for a particular RNA
family, or aligning a few thousand SSU rRNA sequences.  But
\textsc{infernal} remains computationally expensive, and many problems
of interest require the use of a cluster.  The most expensive programs
(\emph{cmcalibrate}, \emph{cmsearch}, and \emph{cmalign}) are
implemented in coarse-grained parallel MPI versions.

%\textsc{infernal} is a sensitive tool for RNA homology search and
%alignment. Up to this point, \textsc{infernal}'s main use has been
%within the Rfam database \citep{Gardner09}, but the heuristic
%acceleration techniques in version 1.0 make other useful applications
%feasible to users with less compute, including RNA annotation of
%prokaryotic genomes, large-scale database searching for an RNA
%family of interest, and megasequence scale alignments of ribosomal
%RNAs.  

%\textsc{Infernal} is designed for (and most useful for) the
%seed profile strategies used by databases such as Pfam and Rfam
%\citep{Finn08,Gardner09}, in which a stable, representative,
%well-annotated ``seed'' alignment of a sequence family is curated, and
%a computational profile of that seed alignment (either a
%\textsc{hmmer} profile HMM in the case of Pfam, or an
%\textsc{infernal} CM in the case of Rfam) is used to identify and
%align additional members of the family. 


%The \emph{cmbuild} program requires as input a structurally annotated
%multiple sequence alignment. The \textsc{infernal} implementation
%currently does not attempt to predict the consensus structure of a
%sequence alignment, nor does it infer an alignment from unaligned
%sequences \emph{de novo}. 

%\textsc{infernal} remains computationally expensive. It generally
%requires the use of a cluster, rather than a single desktop computer,
%for most problems of interest. The most expensive programs
%(\emph{cmcalibrate}, \emph{cmsearch}, and \emph{cmalign}) are
%implemented in coarse-grained parallel MPI versions for use on
%clusters. 


The complete \textsc{infernal} version 1.0 software package, including
documentation and ANSI C source code, may be downloaded from
%\href{http://infernal.janelia.org}. It uses a GNU configure system and
\url{http://infernal.janelia.org}. \textsc{infernal} uses a GNU
configure system and should be portable to any POSIX-compliant
operating system, including Linux and Mac OS/X.
%It is freely licensed under the GNU General Public License,
%version 3.

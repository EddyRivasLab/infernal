\section{Discussion}

The \emph{cmbuild} program requires as input a structurally annotated
multiple sequence alignment. The \textsc{infernal} implementation
currently does not attempt to predict the consensus structure of a
sequence alignment, nor does it infer an alignment from unaligned
sequences \emph{de novo}. It is designed for (and most useful for) the
seed profile strategies used by databases such as Pfam and Rfam
\citep{Finn08,Gardner09}, in which a stable, representative,
well-annotated ``seed'' alignment of a sequence family is curated, and
a computational profile of that seed alignment (either a
\textsc{hmmer} profile HMM in the case of Pfam, or an
\textsc{infernal} CM in the case of Rfam) is used to identify and
align additional members of the family.

\textsc{infernal} remains computationally expensive. It generally
requires the use of a cluster, rather than a single desktop computer,
for most problems of interest. The most expensive programs
(\emph{cmcalibrate}, \emph{cmsearch}, and \emph{cmalign}) are
implemented in coarse-grained parallel MPI versions for use on
clusters. 

The complete \textsc{infernal} version 1.0 software package, including
documentation and ANSI C source code, may be downloaded from
\url{http://infernal.janelia.org}. It uses a GNU configure system and
should be portable to any POSIX-compliant operating system, including
Linux and Mac OS/X. It is freely licensed under the GNU General Public
License, version 3.


\section{Performance}

A published benchmark (independent of our lab) found that
\textsc{infernal} and other CM based methods were the most sensitive
and specific tools for structural RNA homology search among the
several that were tested \citep{Freyhult07}.  Results on a separate
benchmark that we use during \textsc{infernal} development have been
consistent with that conclusion \citep{NawrockiEddy07}.

Figure~1 shows updated results of our internal benchmark comparing
\textsc{infernal} 1.0 to the previous version (0.72) that was
benchmarked in \citep{Freyhult07}, and also to family-pairwise-search
with BLASTN \citep{Altschul97,Grundy98b}.  The sensitivity and
specificity of \textsc{infernal} 1.0 have greatly improved. There have
been three relevant improvements in the implementation: a biased
composition correction to the raw log-odds scores, the use of the full
Inside log-likelihood scores (summed over all alignments) in place of
maximum likelihood alignment scores, and the introduction of
approximate E-value estimates for the scores.

The benchmark dataset used in Figure~1 is an improved version of our
internal Rfam-based benchmark \citep{NawrockiEddy07}. Briefly, this
benchmark was constructed as follows. The sequences of the seed
alignments of 503 Rfam (release 7) families were single linkage
clustered by pairwise sequence identity, and separated into two
clusters such that no sequence in one cluster is more than 60\%
identical to any sequence in the other. The larger of the two clusters
was assigned as the query (preserving their original Rfam alignment
and structure annotation), and the sequences in the smaller cluster
were assigned as true positives in a test set. We required a minimum
of five sequences in the query alignment. 51 Rfam families met these
criteria, yielding 450 test sequences, as described in
\citep{NawrockiEddy07}. We then embedded the 450 test sequences at
random positions in a 10 Mb ``pseudogenome''. Previously we generated
the pseudogenome sequence from a uniform residue frequency
distribution \citep{NawrockiEddy07}. Because base composition biases
in the target sequence database cause the most serious problems in
separating significant CM hits from noise, we improved the realism of
the benchmark by generating the pseudogenome sequence from a 15-state
fully connected hidden Markov model (HMM) trained by Baum-Welch
expectation maximization \citep{Durbin98} on genome sequence data from
a wide variety of species. Each of the 51 query alignments was used to
build a CM and search the pseudogenome, a single list of all hits for
all families were collected and ranked, and true and false hits were
defined (as described in \citep{NawrockiEddy07}), producing the ROC
curves in Figure~1.

\textsc{infernal} searches require a large amount of compute time To
alleviate this, \textsc{infernal} 1.0 implements two rounds of
filters.  When appropriate, the HMM filtering technique described by
Weinberg and Ruzzo \citep{WeinbergRuzzo06} is applied first with
filter thresholds configured by \emph{cmcalibrate} (occasionally a
model with little primary sequence conservation cannot be usefully
accelerated by a primary sequence based filter).  The query-dependent
banded (QDB) CYK search algorithm is used as a second filter with
relatively tight bands ($\beta$=1E-7) \citep{NawrockiEddy07}.  The
final, post-filtering, stage of search with the Inside algorithm also
uses QDBs, but with looser bands ($\beta$=1E-15).  The benchmark in
Figure~1 shows results with and without the filters. The default
filters accelerate similarity search for the benchmark by about
30-fold overall, while sacrificing a small amount of sensitivity. This
makes \textsc{infernal} 1.0 substantially faster than 0.72, although
\textsc{BLAST} is still orders of magnitude faster.  Table~1 shows
specific examples of running times for similarity searches for six RNA
families of various sizes with and without filters. 
%The acceleration
%gained from filters varies widely. In general, the more primary
%sequence conservation in a family, the more effective an HMM filter
%will be, although this is not always the case. 
Further acceleration remains a major goal of \textsc{infernal}
development.

The computational cost of CM alignment with the \emph{cmalign} program
has been a limitation of previous versions of
\textsc{infernal}. Version 1.0 now uses a constrained dynamic
programming approach first developed by Michael Brown \citep{Brown00}
that uses sequence specific bands derived from a first-pass HMM
alignment. This technique offers a dramatic speedup relative to
unconstrained alignment, especially for large RNAs such as small and
large subunit (SSU and LSU) ribosomal RNAs, which can now be aligned
in roughly 1 and 3 seconds per sequence, respectively (Table 1), as
opposed to 12 minutes and 3 hours in previous versions (data not
shown). We expect this to be particularly useful in applications where
many large RNA sequences need to be aligned. One of the main ribosomal
RNA databases, RDP, has recently adopted \textsc{infernal} in its
pipeline \citep{Cole09}.


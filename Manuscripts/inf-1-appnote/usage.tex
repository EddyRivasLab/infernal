
\section{Usage} 

A CM is built from a multiple sequence alignment (or single RNA
sequence) with consensus secondary structure annotation marking which
positions of the alignment are to be scored as single stranded and
which are to be scored as base paired. CMs assign position specific
scores for the four possible residues at single stranded positions and
the sixteen possible base pairs at paired positions, as well as
position specific scores for insertions and deletions. These scores
are log-odds scores derived from the observed counts of residues, base
pairs, insertions and deletions in the input alignment, combined with
prior information derived from structural ribosomal RNA
alignments. Construction and parameterization of CMs have been
described in more detail elsewhere
\citep{Eddy94,infguide03,Eddy02b,NawrockiEddy07}.

\textsc{infernal} is composed of several programs that are used in
combination to build models, search databases, and align putative
homologs, following four basic steps:

\begin{enumerate}
\item Build a CM from an input alignment.

The \emph{cmbuild} program takes as input a structural multiple
RNA alignment in Stockholm format \citep{infguide03} and creates a CM
file that is used by other \textsc{infernal} programs.

\item Calibrate the CM file for similarity search.

Prior to searching databases, parameters for approximate E-value
statistics for a CM should be estimated using the \emph{cmcalibrate}
program. This step is optional and computationally expensive (as shown
in Table~1), but is required to obtain E-values that estimate the
statistical significance of each hit in a database similarity
search. \emph{cmcalibrate} will also determine appropriate HMM filter
thresholds used to accelerate searches without an appreciable loss of
sensitivity, as described in more detail below.

\item Search databases for putative homologs.

The \emph{cmsearch} program takes a CM file as input and searches a
sequence file for high scoring hits to the model. The output of
\emph{cmsearch} includes an alignment of each hit in a BLAST-like
format augmented with structure annotation and (optionally) with
posterior probabilities indicating the confidence in each aligned
position.

\item Align putative homologs to the model.

\emph{cmalign} takes a CM file as input and a target sequence file
containing putative homologs and aligns the full length sequences to
the model, creating a structurally annotated multiple alignment in
Stockholm format.

\end{enumerate}

Some of these steps are unnecessary for some applications. For
example, a user that wants only to generate alignments of previously
defined homologous sequences, such as small subunit ribosomal RNA
sequences (SSU rRNA), would skip the calibration and search steps. 

For similarity search applications, where the goal is to identify new
examples of a family, it is reasonable to iterate these steps, adding
newly found homologs to the alignment and repeating the search as the
detected range of the family expands. Just as with primary sequence
profiles, the ability of CMs to detect remote homologs tends to
increase as the diversity of known sequences in the query alignment
increases.

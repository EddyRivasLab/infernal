
\section{Usage} 

A CM is built from a multiple sequence alignment (or single RNA
sequence) with consensus secondary structure annotation marking which
positions of the alignment are single stranded and which are base
paired. CMs assign position specific scores for the four possible
residues at single stranded positions and the sixteen possible base
pairs at paired positions, as well as position specific scores for
insertions and deletions. These scores are log-odds scores derived
from the observed counts of residues, base pairs, insertions and
deletions in the input alignment, combined with prior information
derived from structural ribosomal RNA alignments. Construction and
parameterization of CMs have been described in more detail elsewhere
\citep{Eddy94,infguide03,Eddy02b,NawrockiEddy07}.

\textsc{infernal} is composed of several programs that are used in
combination to build models, search databases, and align putative
homologs, following four basic steps:

\begin{enumerate}
\item Build a CM from an input alignment with \emph{cmbuild}.

\emph{cmbuild} takes as input a structural multiple
RNA alignment in Stockholm format \citep{infguide03} and creates a CM
file that is used by other \textsc{infernal} programs.

\item Calibrate a CM for similarity search with \emph{cmcalibrate}.

This step is optional and computationally expensive (Table~1), but is
required to obtain E-values that estimate the statistical significance
of each hit in a database search. \emph{cmcalibrate} will
also determine appropriate HMM filter thresholds for accelerating
searches without an appreciable loss of sensitivity. Each model must
only be calibrated once.
%, and can subsequently be used for multiple
%database searches.

\item Search databases for putative homologs with \emph{cmsearch}.

Given a CM file and a target database as input, 
\emph{cmsearch} searches the target database
for high scoring hits to the model and outputs alignments
of each hit in a BLAST-like format augmented with structure
annotation.

\item Align putative homologs to a CM with \emph{cmalign}.

\emph{cmalign} takes a CM file and a file of putative homologs, and
aligns the full length sequences to the model, creating a structurally
annotated multiple alignment in Stockholm format.

\end{enumerate}

%Some steps are unnecessary for some applications. For example, a user
%that wants only to generate alignments of previously defined
%homologous sequences, such as small subunit ribosomal RNA (SSU rRNA)
%sequences, would skip the calibration and search steps.

%For similarity search applications, where the goal is to identify new
%examples of a family, it is reasonable to iterate these steps, adding
%newly found homologs to the alignment and repeating the search as the
%detected range of the family expands. Just as with primary sequence
%profiles, the ability of CMs to detect remote homologs tends to
%increase as the diversity of known sequences in the query alignment
%increases.

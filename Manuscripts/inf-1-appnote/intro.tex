\section{Introduction}

When searching for homologous structural RNAs in sequence databases,
it is desirable to score both primary sequence and secondary
structure conservation. 
%Many tools for integrating and scoring RNA
%sequence and secondary structure have been developed.
%Some implement
%specialized rules for a specific RNA family
%\citep{LoweEddy97,Laslett04,LoweEddy99,Schattner06,Lai03,
%%Lim03,
%Regalia02},
%and others use pattern matching methods and expertly designed query
%patterns \citep{Macke01}. 
The most generally useful tools that integrate sequence and structure
%approaches 
take as input
any RNA (or RNA multiple alignment), and construct an appropriate
statistical scoring system that allows quantitative ranking of
putative homologs in a sequence database
\citep{Gautheret01,ZhangBafna05,Huang08}.  Stochastic context-free
grammars (SCFGs) provide a natural statistical framework for combining
sequence and (non-pseudoknotted) secondary structure conservation
information in a single consistent scoring system
\citep{Sakakibara94c,Eddy94,Brown00,Durbin98}.

\begin{comment}
When searching for homologous structural RNAs in sequence databases,
it is desirable to score both primary sequence and RNA secondary
structure conservation. Many tools for integrating and scoring RNA
sequence and secondary structure have been developed. Some implement
specialized rules for a specific RNA family
\citep{LoweEddy97,Laslett04,LoweEddy99,Schattner06,Lai03,
%Lim03,
Regalia02},
and others use pattern matching methods and expertly designed query
patterns \citep{Macke01}. 
The most general approaches take as input
any RNA (or RNA multiple alignment), and construct an appropriate
statistical scoring system that allows quantitative ranking of
putative homologs in a target sequence database
\citep{Gautheret01,ZhangBafna05,Huang08}.  Stochastic context-free
grammars (SCFGs) provide a natural statistical framework for combining
sequence and (non-pseudoknotted) secondary structure conservation
information in a single consistent scoring system
\citep{Sakakibara94c,Eddy94,Brown00,Durbin98}.
\end{comment}


%When searching for homologous structural RNAs in sequence databases,
%it is desirable to score both primary sequence and RNA secondary
%structure conservation. Several tools have been developed which take
%different approaches to how RNA sequence and secondary structure
%constraints should be integrated and scored. Some tools implement
%specialized rules for a specific RNA family, such as tRNAs
%\citep{LoweEddy97,Laslett04}, snoRNAs \citep{LoweEddy99,Schattner06},
%microRNAs \citep{Lai03,Lim03}, or SRP RNAs \citep{Lai03,Lim03}. Some
%approaches use pattern matching methods and expertly designed query
%patterns \citep{Macke01}. The most general approaches take as input
%any RNA (or RNA multiple alignment), and construct an appropriate
%statistical scoring system that allows quantitative ranking of
%putative homologs in a target sequence database
%\citep{Gautheret01,ZhangBafna05,Huang08}.  Stochastic context-free
%grammars (SCFGs) provide a natural statistical framework for combining
%sequence and (non-pseudoknotted) secondary structure conservation
%information in a single consistent scoring system
%\citep{Sakakibara94c,Eddy94,Brown00,Durbin98}.

Here, we announce the 1.0 release of \textsc{infernal}, an
implementation of a general SCFG-based approach for RNA database
searches and multiple alignment. \textsc{infernal} builds consensus
RNA profiles called \emph{covariance models} (CMs), a special case of
SCFGs designed for modeling RNA consensus sequence and structure. It
uses CMs to search nucleic acid sequence databases for homologous
RNAs, or to create new sequence and structure-based multiple %sequence
alignments. One use of \textsc{infernal} is to annotate RNAs in
genomes in conjunction with the \textsc{Rfam} database
\citep{Gardner09}, which contains hundreds of RNA families.
\textsc{Rfam} follows a seed profile strategy, in which a
well-annotated ``seed'' alignment of each family is curated, and a CM
built from that seed alignment is used to identify and align
additional members of the family.  \textsc{infernal} has been in use
since 2002, but 1.0 is the first version that we consider to be a
reasonably complete production tool. It now includes E-value estimates
for the statistical significance of database hits, and heuristic
acceleration algorithms for both database searches and multiple
alignment that allow \textsc{infernal} to be deployed in a variety of
real RNA analysis tasks with manageable (albeit high) computational
requirements.


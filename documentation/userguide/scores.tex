\section{How \texttt{cmsearch} scores alignments and determines significance}

The \prog{cmsearch} program gives you a ranked list of hits in a
sequence database.  Which ones are likely to be true homologues and
which ones are likely to be nonhomologous to your query CM?

\software{Infernal} gives you at least two scoring criteria to judge
by: the \software{Infernal} raw score, and an E-value. Additionally,
Rfam models carry a third set of criteria: three expert-calibrated raw
score cutoffs that the Rfam database maintainers set. How should you
interpret all this information?
\subsection{Executive summary}

\begin{itemize}
\item The best criterion of statistical significance is the E-value.
The E-value is calculated from the bit score. It tells you how many
false positives you would have expected to see at or above this bit
score. Therefore a low E-value is best; an E-value of 0.1, for
instance, means that there's only a 10\% chance that you would've seen
a hit this good in a target database of random sequences of the same
size as the one you're searching. {\em
Typically, I trust the results of searches at about E=0.1 and below,
and I examine the hits manually down to E=10 or so.}  However, be
alert; \software{Infernal} E-values are not perfect. MORE HERE.

\item \software{Infernal} bit scores are a stricter criterion: they
  reflect whether the sequence is a better match to the profile model
  (positive score) or to the null model of nonhomologous sequences
  (negative score).  A bit score above $\log_2$ of the size of the
  target database is likely to be a true
  homologue. For a 10 Mb genome, this rule-of-thumb number is on
  the order of 24 bits (remember a 10 Mb genome is really 20 Mb when
  you search both strands).  Whereas the E-value measures how
  statistically significant the bit score is, the bit score itself is
  telling you how well the sequence matches your model. Because these
  things should be strongly correlated, usually, true homologues will
  have both a good bit score and a good E-value. However, sometimes
  (and these are the interesting cases), you will find remote
  homologues which do not match the model well (and so do not have
  good bit scores -- possibly even negative), but which nonetheless
  have significant E-values, indicating that the bit score, though
  ``bad'', is still better than you would've expected by chance, so it
  is suggestive of homology. However, as above, be alert; it is
  possible to get artifactually high bit scores simply because of
  highly biased residue composition.

\item For Rfam CMs, you can also examine three other numbers that
represent bit score thresholds: a TC (trusted cutoff) score, a GA
(gathering) score, and a NC (noise cutoff) score. The meaning of
these numbers is described below.
\end{itemize}

\begin{srefaq}{What does it mean when I have a negative bit score,
but a good E-value?} The negative bit score means that the sequence is
not a good match to the model. The good E-value means that it's still
a better score than you would've expected from a random sequence. The
usual interpretation is that the sequence is homologous to the
sequence family modeled by the CM, but it's not ``within'' the family
- it's a distant homologue of some sort. This happens most often with
CMs built from ``tight'' families of high sequence identity, aligned
to remote homologues outside the family. For example, a bacterial SRP
CM aligned to a eukaryotic SRP may show this behavior - the bit
score says the sequence isn't a bacterial SRP (correct) but the E-value says
it is significantly related to the bacterial SRP family (also correct).
\end{srefaq}

\subsection{In more detail: \software{Infernal} bit scores}

The bit score is a log-odds score in log base two (thus, in units of
{\em bits}). Specifically, it is:

\[
	S = \log_2 \frac {P( \mbox{seq} | \mbox{CM})} { P (\mbox{seq} |
	\mbox{null})}.
\]

$P( \mbox{seq} | \mbox{CM})$ is the probability of the target
sequence according to your CM. $ P (\mbox{seq} | \mbox{null}) $ is
the probability of the target sequence given a ``null hypothesis''
model of the statistics of random sequence. In \software{Infernal}, this null model
is a simple one-state CM that says that random sequences are i.i.d.
sequences with a specific residue composition, which by default is
equiprobable across the four RNA nucleotides ($P(A) = P(C) = P(G) =
P(U) = 0.25$). This ``null model distribution'' is part of the CM save
file, and it can be altered when you run \prog{cmbuild}.

Thus, a positive score means the CM is a better model of the target
sequence than the null model is (e.g. the CM gives a higher
probability).

You can specify the E-value or bit score cutoff to the \prog{cmsearch}
program. By default, for calibrated models (see below) the E-value
cutoff is set as 0.1, and for non-calibrated models (for which
E-values are not available) the cutoff is set as a bit score of 0.0
bits. (This is discussed in more detail in section 6). Alternatively,
for a specific sensible use cutoffs, read about how the Rfam
TC/NC/GA cutoffs are set and used (below).

\subsection{In more detail: \software{Infernal} E-values}

The E-value is the expected number of false positives with scores at
least as high as your hit.

Unlike the raw score, the E-value is dependent on the size of the
database you search. If you detect a hit of length 100 residues 
with an E-value of 0.1 in a search of a sequence database of size 100
Mb, then you happen to re-score the sequence all by itself, you will
get an E-value 1 million times better. The E-value is quite literally
the expected number of false positives at this raw score; the larger
the database you search, the greater the number of expected false
positives.

\begin{srefaq}{Why do I get a different E-value when I search
against a file containing my sequence, than I got when I searched the
database?} See above. This behavior is shared with BLAST and FASTA
P-value and E-values, so it should not be unfamiliar to most users.
However, it can cause consternation: a related phenomenon is that a
hit that is marginally significant this year may no longer be
significant next year, when the database is twice as large. 
\end{srefaq}

To calculate E-values, models must be calibrated with the
\prog{cmcalibrate} program. Unfortunately, \prog{cmcalibrate} is
painfully slow. We're working on making it faster, but for now, it is
highly recommended that you spend the compute time to calibrate your
models before searching with them. Not only will calibrated models
report E-values, but they are often much faster at searching. This is
because, in addition to calibrating E-value statistics, \prog{cmcalibrate}
also determines appropriate HMM filter cutoffs to use for each model, as
explained in section 6.

\prog{cmcalibrate} writes several parameters into your CM file on
lines with labels starting with ``E-''. Among these parameters are the
$\mu$ (location) and $\lambda$ (scale) parameters of an exponential
tail that best fits a histogram of scores calculated on randomly
generated sequences. You only need to do this calibration once
for a given CM. All the Rfam CMs come pre-calibrated.

\subsubsection{fitting exponential tails to \software{Infernal} score histograms}

The \prog{cmcalibrate} program fits exponential tails to scores of the
model against random sequences. When \prog{cmsearch} is run and hits
are found, the exponential tail parameters are used to calculate
E-value for the hits. \prog{cmsearch} implements different search
algorithms and allows searching with both local and glocally
configured models. Importantly, the configuration and algorithm used
affect the scores reported, so \prog{cmcalibrate} must fit separate
exponential tails for each.  In addition, separate
exponential tails are fit for three possible ranges, or partitions of
the possible GC contents of the target database, 0-39\%, 40-59\% and
60-100\%.  For example, the scores for the histogram that is fit to an
exponential tail for the 0-39\% GC region are collected by searching
random sequence with GC content between 0 and 39\%. In total,
\prog{cmcalibrate} fits 24 different exponential tail
distributions. These are listed by the program as it proceeds through
each of the 24 ``stages''.  Because \prog{cmcalibrate} is so slow, it
has a \prog{--forecast} option which will list each stage and predict
it's required running time. Let's examine the output with this option
for the \prog{my.cm} model from the tutorial:

\user{cmcalibrate --exp-gc ~/infernal/src/1Mb.fa --exp-pfile ~/infernal/src/p4060.pfile --forecast 1 my.cm}

\begin{sreoutput}
# Forecasting time for 1 processor(s) to calibrate CM 1: tobuild-1
#
# stage     mod  cfg  alg  part  ps  pe expL (Mb)   filN predicted time
# --------  ---  ---  ---  ---- --- --- --------- ------ --------------
  exp tail  hmm  glc  vit     1   0  39     10.16      -       00:01:17
  exp tail  hmm  glc  vit     2  40  59     10.16      -       00:01:17
  exp tail  hmm  glc  vit     3  60 100     10.16      -       00:01:17
  exp tail  hmm  glc  fwd     1   0  39     10.16      -       00:02:45
  exp tail  hmm  glc  fwd     2  40  59     10.16      -       00:02:45
  exp tail  hmm  glc  fwd     3  60 100     10.16      -       00:02:45
  exp tail   cm  glc  cyk     1   0  39      1.00      -       00:06:07
  exp tail   cm  glc  cyk     2  40  59      1.00      -       00:06:07
  exp tail   cm  glc  cyk     3  60 100      1.00      -       00:06:07
  exp tail   cm  glc  ins     1   0  39      1.00      -       00:21:09
  exp tail   cm  glc  ins     2  40  59      1.00      -       00:21:09
  exp tail   cm  glc  ins     3  60 100      1.00      -       00:21:09
  filter      -  glc    -     -   -   -         -  10000       00:02:51
  exp tail  hmm  loc  vit     1   0  39     10.16      -       00:01:25
  exp tail  hmm  loc  vit     2  40  59     10.16      -       00:01:25
  exp tail  hmm  loc  vit     3  60 100     10.16      -       00:01:25
  exp tail  hmm  loc  fwd     1   0  39     10.16      -       00:03:25
  exp tail  hmm  loc  fwd     2  40  59     10.16      -       00:03:25
  exp tail  hmm  loc  fwd     3  60 100     10.16      -       00:03:25
  exp tail   cm  loc  cyk     1   0  39      1.00      -       00:06:24
  exp tail   cm  loc  cyk     2  40  59      1.00      -       00:06:24
  exp tail   cm  loc  cyk     3  60 100      1.00      -       00:06:24
  exp tail   cm  loc  ins     1   0  39      1.00      -       00:24:10
  exp tail   cm  loc  ins     2  40  59      1.00      -       00:24:10
  exp tail   cm  loc  ins     3  60 100      1.00      -       00:24:10
  filter      -  loc    -     -   -   -         -  10000       00:06:35
# --------  ---  ---  ---  ---- --- --- ---------  ----- --------------
# all         -    -    -     -   -   -         -      -       03:29:47
\end{sreoutput}

Let's go through what each column of the output is telling us:

\begin{wideitem}
\item[\emprog{stage}] what type of stage this row pertains to, either
  ``exp tail'' for exponential tail fitting, or ``filter'' for filter
  threshold calculation. For now we're just focusing on the exponential tail
  stages. 

\item[\emprog{mod}] the model we're fitting an exponential tail for,
  either the CM or a CP9 HMM. We use the HMM exponential tails during
  HMM filtering for faster searches.

\item[\emprog{cfg}] the configuration of the model for this stage,
  either ``glc'' for glocal or ``loc'' for local. \prog{cmcalibrate}
  has to calibrate exponential tails \emph{separately} for glocal and
  local modes.

\item[\emprog{alg}] the search algorithm for this stage. There are two
  algorithms that need to be separately calibrated for each the CM and
  HMM. These are explained in more detail in section 6.

\item[\emprog{part, ps} and \emprog{pe}] which partition of GC
  content this stage is for. ``part'' is the partition index, ``ps''
  is the lowest GC content in the partition and ``pe'' is the highest
  GC content. A different exponential tail is fit for
  each partition. 

\item[\emprog{expL}] the length of random sequence we'll search for
  this stage. For CM stages, this length is 1,000,000 residues (1 Mb)
  by default. For HMM stages, this length is is the minimum
  of 10 Mb and a length $x$, where $x$ is the length that will cause
  this HMM stage to require 10\% as much compute time as a CM
  calibration stage. The reasoning behind this is that longer sequence
  lengths tend to result in more accurate E-values.
%  Mb, but can be more if the program calculates that searching a
%  longer sequence with the HMM (which will lead to more accurate
%  E-values) will not increase the calibration time of this stage to more
%  than 10\% the total calibration time.

\item[\emprog{filN}] the number of random sequences that will be
  searched for the HMM filter threshold calculation. This isn't
  relevant now; see section 6 for more on HMM filter threshold
  calculation. 

\item[\emprog{predicted time}] the predicted run time of this stage. 

\end{wideitem}

As you can see, \prog{cmcalibrate} will fit separate exponential tails
for each combination of model and algorithm (4 choices), configuration
(2 choices), and partition (3 choices), that's $4*2*3=24$ exponential
tails. We explained the difference between local versus glocal
configuration in the tutorial, and described partitions briefly above;
now we'll discuss different search algorithms. 

\subsection{In more detail: different search algorithms in \software{Infernal}}
\prog{cmsearch} implements four several different search 
algorithms, Inside and CYK for profile SCFG search with CMs, and
Forward and Viterbi for profile HMM search with CP9 HMMs.
These algorithms
differ in the meaning of the score that they calculate. Above we
oversimplified an \software{Infernal} bit score as:
\[
	S = \log_2 \frac {P( \mbox{seq} | \mbox{CM})} { P (\mbox{seq} |
	\mbox{null})}.
\]

This is actually only technically correct for the Inside
algorithm. The other three algorithms listed above are calculating
something slightly different in the \emph{numerator} (only) of the
above equation, as follows: 

\begin{wideitem}
\item[\em{Inside}]  $P(\mbox{seq}      | \mbox{CM})$
\item[\em{CYK}]     $P(\mbox{seq},\pi  | \mbox{CM})$
\item[\em{Forward}] $P(\mbox{seq}      | \mbox{HMM})$
\item[\em{Viterbi}] $P(\mbox{seq},\pi  | \mbox{HMM})$
\end{wideitem}

Here, $\pi$ represents the optimal (most probable) parse, or
alignment, of the sequence to the model. So, the CYK and Viterbi
algorithms report the log-odds score for the optimal alignment 
of the sequence given the model (either CM or HMM respectively), while
the Inside and Forward algorithms report the log odds score for the
sequence \emph{summed over all possible alignments} of the sequence to
the model. 

From a probabilistic inference standpoint, the Inside and Forward
scores are giving us what we want to know, the log-odds score that the
sequence we're looking at is a homolog of the family we're
modelling. The specific alignment $\pi$ to the model is a nuisance
variable that is appropriately integrated out. Not only are Inside and
Forward better in theory, but empirically, benchmarks show that
they're more sensitive than CYK and Viterbi. So, why use CYK and
Viterbi at all?  Actually, we rarely use Viterbi really, it's
only implemented for testing purposes (it's okay; calibrating Viterbi
E-values takes a very small fraction of the running time of
\prog{cmcalibrate}, you can see for yourself in the previous
example). CYK \emph{is} used, but mainly as a filter in
\prog{cmsearch}. CYK makes a good filter for two reasons. First, it can be
more efficiently implemented than the Inside
algorithm. \software{Infernal}'s current implementation of CYK is
about three times faster than Inside (you can see this in the
predicted run times in the \prog{--forecast} example). Secondly,
high scoring CYK hits, the ones where interested in, tend to have a
single well-defined alignment and consequently approximate Inside
scores well. Combined, these two features mean we can safely and
effectively filter with CYK, and have the speed of CYK and sensitivity
of Inside. You can read more about how we use CYK as a filter in
\prog{cmsearch} in section 6.


\begin{srefaq}{Why are you talking about HMM algorithms. I thought CMs
    were more appropriate for structural RNA sequence analysis than
    HMMs, isn't that the whole reason you've developed \textsc{Infernal}?} 
  We've implemented HMMs in \textsc{Infernal} to use to accelerate CM
    algorithms. As described in section 6, HMMs are used as filters to
    speedup database searches with \prog{cmsearch}. They're also used to develop
    constraints for CM alignment with \prog{cmalign}.  
\end{srefaq}

%This is the main reason \prog{cmcalibrate} is taking so
%long. Trust us, we'd love to be able to cut down on these stages and
%accelerate calibration, but currently if we cut out any stages, we get
%inadequate performance. We're working on it though.

\begin{srefaq}{Why is \prog{cmcalibrate} so slow?} The
    necessity of doing 24 separate exponential tail fitting stages
    (see above), and the fact that CM search algorithms scale more
    than $LN^2$ with sequence length $L$ and model consensus length
    $N$ make the program excruciatingly slow. Notice that some of the
    24 stages are much faster than others. The 6 Inside stages usually
    take a large majority of the total run time. You might argue this
    means we should get rid of Inside, but it's the most sensitive
    algorithm we have; and sensitivity trumps speed in our design
    goals.
\end{srefaq}

%\subsubsection{Partitions of GC content}
%A major problem facing CM homology search is heterogeneity of the
%target database. Differences in local GC content can have a profound
%effect on CM bit scores of non-homologous sequences. Imagine a
%sequence of 80\% A's and U's, the probability of a stem of AU base
%pairs ``forming'' and getting a high score to a structural RNA model 
%is increased relative to a 50\% AU sequence, simply because there are 
%more A's and U's. Empirically, when searching databases, using a
%single exponential tail for E-value statistics proved inadequate. 

\subsection{Accuracy of E-values}

NEED TO WRITE THIS WHEN FINAL E-VALUE IMPLEMENTATION IS DONE.

\begin{comment}
First the good news. The extreme value distribution fitting is done
with a rather robust and personally satisfying chunk of maximum
likelihood code (see \prog{histogram.c} in the codebase). The ML
approach was suggested to me by Stephen Altschul, who directed me to a
lovely textbook by Lawless \cite{Lawless82}.  A brief technical report
on \software{Infernal}'s EVD fitting is available at
\htmladdnormallink{ftp://ftp.genetics.wustl.edu/pub/eddy/papers/evd.pdf}
{ftp://ftp.genetics.wustl.edu/pub/eddy/papers/evd.pdf}.  Any EVD
fitting code that is relying on linear regression fits to log-log
plots is bound to be less accurate, judging from my tests.

However, the down side is that most profile CM scores don't fit the
extreme value distribution well. Fully local alignments (models built
with \prog{hmmbuild -f} or \prog{hmmbuild -s} fit the EVD fairly well,
as expected for Smith/Waterman style alignments, for the same reasons
that gapped BLAST or FASTA alignment scores have been shown to be
approximately EVD-distributed. By default, though, \prog{hmmbuild}
makes models for ``glocal'' alignments which are global with respect
to the model, and multi-hit-local with respect to the sequence.  Also
called ``profile scores'' by Waterman, this sort of alignment score is
known not to be EVD-distributed \cite{GoldsteinWaterman94}.

Nonetheless, empirically, the tail of the distribution around E=1 is
falling off more or less like an EVD, and this is the region of
interest. \software{Infernal} does a ``censored EVD fit'' from the peak of the
distribution down to the right tail, and does not include the data to
the left of the peak in its fit. This produces a reasonable fit in the
important region of the scores. Empirically, \software{Infernal} E-values tend to be
accurate in the critical region (E~1), despite the lack of
mathematical foundation.

The end of \prog{cmsearch} output shows you the observed histogram,
and (if the model is calibrated) an expected curve according to the
EVD fit. You should observe that the relevant tail (around E=1) is
more or less well fitted. The bulk of the distribution, particularly
at lower scores, is usually poorly fitted, but this is not the area of
interest.
\end{comment}

\subsection{In more detail: Rfam TC/NC/GA cutoffs}

When a Rfam model is built, the Rfam curation team keeps track of
scores of every hit in a large nonredundant database. They record
three types of score cutoffs on Rfam CM files:

\begin{wideitem}
\item[GA (gathering cutoff)]: the score used as cutoff in
constructing Rfam. All domains that are in a sequence satisfying the
GA bit score will be included in the Rfam ``full alignment''.

\item[TC (trusted cutoff)]: the scores of the lowest-scoring hit(s)
that were included as true member(s) of the Rfam family. Hits above
the TC score are ``within'' the Rfam family and almost certainly
members.

\item[NC (noise cutoff)]: the score of the highest-scoring hit(s) that
were \textit{not} included as true members of the Rfam family, because
they were considered to be the top of the noise.  Hits above the NC
cutoff are above the top scoring noise in the Rfam NR database search,
so are likely homologues, but not as trustworthy as hits over the GA
or TC cutoffs.
\end{wideitem}

In order of increasing conservativeness, the cutoffs rank: NC, GA, and
TC.

The GA cutoff, being the actual cutoff used in constructing Rfam,
are a very good choice to use to collate large-scale automated data,
like counting RNA family members in a sequenced genome.

The TC and NC cutoffs are less useful, and only really there as
documentation of Rfam construction. In general, the TC cutoff would
be a extremely conservative cutoff to use in a database search, more
conservative than GA. The NC cutoff is less conservative than GA.

Why use GA (or the other cutoffs) instead of the E-value? Rfam
artificially imposes a ``flat'', nonhierarchical structure on RNA
sequence space.  Rfam asserts that no Rfam family is related to any
other Rfam family. This is obvious nonsense: many Rfam families are in
fact homologous.  The different SRP families are one example; the
different RNaseP families are another. \software{Infernal} often
detect significant relationships between families that Rfam chooses to
suppress. In these cases, the Rfam GA cutoff will be elevated to
artifically separate two homologous but distantly related subgroups of
the same structural superfamily.

\begin{srefaq}{Why isn't sequence X included in a Rfam full alignment?
It has a significant score!} For the reasons above, the sequences in
Rfam full alignments are harvested using curated GA thresholds, rather
than using score or E-value thresholds. Please don't go writing a
paper that claims CMs don't detect some similarity until you've done
the experiment with CMs (and E-values) instead of just looking at
curated Rfam classifications.
\end{srefaq}

The mechanism that \software{Infernal} uses to incorporate up these cutoffs is
general: Stockholm format multiple sequence alignments can carry
appropriate TC, NC, and GA markup lines. This means that you can use a
Rfam-like cutoff system if you like, just by adding the appropriate
Stockholm markup to your collection of alignments. When these numbers
are available in the CM, \prog{cmsearch} provides options
for setting search cutoffs to GA, TC, or NC automatically (these options
are \prog{--ga, --tc} and \prog{--nc}).

\subsubsection{Predicting running times for searches with Rfam cutoffs}
The next section explains how \prog{cmcalibrate} determines HMM filter
thresholds to use to accelerate \prog{cmsearch}. These filter
thresholds are dependent on the final threshold used in
\prog{cmsearch}, in general, the stricter the final threshold the
stricter the filter threshold and the greater the acceleration from
the filter. Because the Rfam GA/NC/TC cutoffs are relatively strict it
is often possible to achieve large speedups of up to 100-fold or more
when they're used as the final threshold. Section 6 explains this in
more detail, and shows an example of predicting the running time of
filtered searches with Rfam cutoffs using the \prog{cmstat} program.




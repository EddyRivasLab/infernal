\section{Installation}

\subsection{Quick installation instructions from source}

Download the source tarball (\prog{infernal.tar.gz}) from 
\htmladdnormallink{ftp.genetics.wustl.edu/pub/eddy/software/infernal/}
                  {ftp://ftp.genetics.wustl.edu/pub/eddy/software/infernal/}
or 
\htmladdnormallink{www.genetics.wustl.edu/eddy/infernal/}
                  {http://www.genetics.wustl.edu/eddy/infernal/}.


Unpack, configure, make, and test the software:
  
\begin{center}\begin{minipage}{5.5in}
\user{tar xvf infernal.tar.gz}
\user{cd infernal}
\user{./configure}
\user{make}
\user{make check}
\end{minipage}\end{center}

All the tests should pass.

The programs are in the \prog{src/} subdirectory. The user's guide
(this document) is in the \prog{documentation/userguide}
subdirectory. The man pages are in the \prog{documentation/manpages}
subdirectory. You can manually move or copy all of these to
appropriate locations if you want. You will want the programs to be in
your \$PATH.

More complete instructions follow, including how to install the
package automatically with \prog{make install}.

\subsection{Detailed installation notes}

\subsection{Supported platforms}

\subsection{Configuration options}

\package{Infernal} uses a GNU configure script to automatically
configure the source code for your platform. Normally 

\begin{sreitems}
\item[--enable-lfs] 
\item[



\section{Installation}

\subsection{Quick installation instructions}

Download the source tarball (\prog{infernal.tar.gz}) from 
\htmladdnormallink{ftp://selab.janelia.org/pub/software/infernal/}
                  {ftp://selab.janelia.org/pub/software/infernal/}
or \\
\htmladdnormallink{http://infernal.janelia.org}
                  {http://infernal.janelia.org}

Unpack the software:

\user{tar xvf infernal.tar.gz}

Go into the newly created top-level directory (named either
\prog{infernal}, or \prog{infernal-xx} where \prog{xx} is a release
number:

\user{cd infernal}

Configure for your system, and build the programs:

\user{./configure}\\
\user{make}

Run the automated testsuite. This is optional. All these tests should
pass:

\user{make check}

The programs are now in the \prog{src/} subdirectory. The user's guide
(this document) is in the \\ 
\prog{documentation/userguide}
subdirectory. The man pages are in the \prog{documentation/manpages}
subdirectory. You can manually move or copy all of these to
appropriate locations if you want. You will want the programs to be in
your \$PATH. 

Optionally, you can install the man pages and programs in system-wide
directories. If you are happy with the default (programs in
\prog{/usr/local/bin/} and man pages in \prog{/usr/local/man/man1}),
do:

\user{make install}

That's all.  More complete instructions follow, including how to
change the default installation directories for \prog{make install}.

\subsection{More detailed installation notes}

\software{Infernal} is distributed as ANSI C source code.
It is designed to be built and used on UNIX platforms. It is developed
on Intel GNU/Linux systems, and intermittently tested on a variety of
vendor-donated UNIX platforms including Sun/Solaris, HP/UX, Digital
Tru64, Silicon Graphics IRIX, IBM/AIX, and Intel/FreeBSD. It is not
currently tested on either Microsoft Windows or Apple OS/X. It should
be possible to build it on any platform with an ANSI C compiler. The
software itself is vanilla POSIX-compliant ANSI C. You may need to
work around the configuration scripts and Makefiles to get it built on
a non-UNIX platform.

The GNU configure script that comes with \software{Infernal} has a
number of options. You can see them all by doing:

\user{./configure --help}

All customizations can and should be done at the \prog{./configure}
command line, unless you're a guru delving into the details of the
source code.

\subsubsection{setting installation targets}

The most important options are those that let you set the installation
directories for \prog{make install} to be appropriate to your system.
What you need to know is that \software{Infernal} installs only two
types of files: programs and man pages. It installs the programs in
\prog{--bindir} (which defaults to \prog{/usr/local/bin}), and the man pages in the
\prog{man1} subdirectory of \prog{--mandir} (default
\prog{/usr/local/man}). Thus, say you want \prog{make install} to install programs
in \prog{/usr/bioprogs/bin/} and man pages in
\prog{/usr/share/man/man1}; you would configure with:

\user{./configure --mandir=/usr/share/man --bindir=/usr/bioprogs/bin}

That's really all you need to know, since \software{Infernal} installs
so few files. But just so you know; GNU configure is very flexible,
and has shortcuts that accomodates several standard conventions for
where programs get installed. One common strategy is to install all
files under one directory, like the default \prog{/usr/local}. To
change this prefix to something else, say \prog{/usr/mylocal/}
(so that programs go in \prog{/usr/mylocal/bin} and man pages in
\prog{/usr/mylocal/man/man1}, you can use the \prog{--prefix}
option:

\user{./configure --prefix=/usr/mylocal}

Another common strategy (especially in multiplatform environments) is
to put programs in an architecture-specific directory like
\prog{/usr/share/Linux/bin} while keeping man pages in a shared,
architecture-independent directory like \prog{/usr/share/man/man1}.
GNU configure uses \prog{--exec-prefix} to set the path to
architecture dependent files; normally it defaults to being the same
as \prog{--prefix}. You could change this, for example, by:

\user{./configure --prefix=/usr/share --exec-prefix=/usr/share/Linux/}\\

In summary, a complete list of the \prog{./configure} installation
options that affect \software{Infernal}:

\begin{tabular}{lll}
Option                       &   Meaning                       & Default\\ \hline
\prog{--prefix=PREFIX}       & architecture independent files  & \prog{/usr/local/} \\
\prog{--exec-prefix=EPREFIX} & architecture dependent files    & PREFIX\\
\prog{--bindir=DIR}          & programs                        & EPREFIX/bin/\\
\prog{--mandir=DIR}          & man pages                       & PREFIX/man/\\
\end{tabular}


\subsubsection{setting compiler and compiler flags}

By default, \prog{configure} searches first for the GNU C compiler
\prog{gcc}, and if that is not found, for a compiler called \prog{cc}. 
This can be overridden by specifying your compiler with the \prog{CC}
environment variable.

By default, the compiler's optimization flags are set to
\prog{-g -O2} for \prog{gcc}, or \prog{-g} for other compilers.
This can be overridden by specifying optimization flags with the
\prog{CFLAGS} environment variable. 

For example, to use an Intel C compiler in
\prog{/usr/intel/ia32/bin/icc} with 
optimization flags \prog{-O3 -ipo}, you would do:

\user{env CC=/usr/intel/ia32/bin/icc CFLAGS="-O3 -ipo" ./configure}

which is the one-line shorthand for:

\user{setenv CC     /usr/intel/ia32/bin/icc}\\
\user{setenv CFLAGS "-O3 -ipo"}\\
\user{./configure}

If you are using a non-GNU compiler, you will almost certainly want to
set \prog{CFLAGS} to some sensible optimization flags for your
platform and compiler. The \prog{-g} default generated unoptimized
code. At a minimum, turn on your compiler's default optimizations with
\prog{CFLAGS=-O}.

\subsubsection{turning on Large File Support (LFS)}

\software{Infernal} has one optional feature: support for Large File
System (LFS) extensions that allow programs to access files larger
than 2 GB. LFS is rapidly becoming standard, but not yet standard
enough to be default. If you do anything with Genbank files or large
genome files (like the 3 GB human genome), you will need LFS support.
LFS is enabled with the \prog{--enable-lfs} option:

\user{./configure --enable-lfs}

\subsubsection{installing rigorous filters}

\software{Infernal} includes programs by Zasha Weinberg that implement
rigorous filtering.  This software requires a C++ compiler, and also
relies on an external library, \software{CFSQP}, that is not included
in this distribution, but can be obtained by request from
\htmladdnormallink{http://www.aemdesign.com/}{http://www.aemdesign.com/}.
To build the executables, include these two options to configure:

\user{./configure --with-rigfilters --with-cfsqp=/path/to/cfsqp}

\subsubsection{installing Message Passing Interface (MPI) programs}

\software{Infernal} includes two programs \prog{mpi-cmsearch} and
\prog{mpi-cmalign} that use MPI parallelization. You have the option
of compiling these two executables, if you have MPI installed. (We use
LAM MPI here, but alternative MPI libraries should also work.)  To
enable MPI and compile these two additional executables, add
\prog{--enable-mpi} to the configuration command:

\user{./configure --enable-mpi}

A \prog{make install} following a \prog{./configure --enable-mpi} will
install the non-MPI programs (\prog{cmalign}, \prog{cmbuild},
\prog{cmemit}, \prog{cmscore}, and \prog{cmsearch} as well as
\prog{mpi-cmsearch} and \prog{mpi-cmalign}.

\prog{mpi-cmsearch} and \prog{mpi-cmalign} have the same command line
arguments as \prog{cmsearch} and \prog{cmalign}, but you must run them
in an MPI environment with \prog{mpirun} or \prog{mpiexec}, for
instance (in our LAM environment) with:

\user{mpirun C mpi-cmsearch query.cm target.fa}

\subsection{Example configuration}

The Intel GNU/Linux version installed at Janelia Farm is configured as
follows:

\user{env CFLAGS="-O3" ./configure --enable-mpi --enable-lfs --prefix=/usr/local/infernal-xx}






\section{Installation}

\subsection{Quick installation instructions}

Download the source tarball (\prog{infernal.tar.gz}) from 
\url{ftp://selab.janelia.org/pub/software/infernal/}

or \\
\url{http://infernal.janelia.org}

Unpack the software:

\user{tar xvf infernal.tar.gz}

Go into the newly created top-level directory (named either
\prog{infernal}, or \prog{infernal-xx} where \prog{xx} is a release
number):

\user{cd infernal}

Configure for your system, and build the programs:

\user{./configure}\\
\user{make}

Run the automated testsuite. This is optional. All these tests should
pass:

\user{make check}

The programs are now in the \prog{src/} subdirectory. The user's guide
(this document) is in the \\ 
\prog{documentation/userguide}
subdirectory. The man pages are in the \prog{documentation/manpages}
subdirectory. You can manually move or copy all of these to
appropriate locations if you want. You will want the programs to be in
your \$PATH. 

Optionally, you can install the man pages and programs in system-wide
directories. If you are happy with the default (programs in
\prog{/usr/local/bin/} and man pages in \prog{/usr/local/man/man1}),
do:

\user{make install}

That's all.  More complete instructions follow, including how to
change the default installation directories for \prog{make install}.

\subsection{More detailed installation notes}

\software{infernal} is distributed as ANSI C source code.  It is
designed to be built and used on UNIX platforms. It is developed on
Intel GNU/Linux systems, and intermittently tested on a variety of
other UNIX platforms. It is not currently tested on either Microsoft
Windows or Apple OS/X, but it should work there; it should be possible
to build it on any platform with an ANSI C compiler. The software
itself is vanilla POSIX-compliant ANSI C. You may need to work around
the configuration scripts and Makefiles to get it built on a non-UNIX
platform.

The GNU configure script that comes with \software{infernal} has a
number of options. You can see them all by doing:

\user{./configure --help}

All customizations can and should be done at the \prog{./configure}
command line, unless you're a guru delving into the details of the
source code.

\subsubsection{setting installation targets}

The most important options are those that let you set the installation
directories for \prog{make install} to be appropriate to your system.
What you need to know is that \software{infernal} installs only two
types of files: programs and man pages. It installs the programs in
\prog{--bindir} (which defaults to \prog{/usr/local/bin}), and the man pages in the
\prog{man1} subdirectory of \prog{--mandir} (default
\prog{/usr/local/man}). Thus, say you want \prog{make install} to install programs
in \prog{/usr/bioprogs/bin/} and man pages in
\prog{/usr/share/man/man1}; you would configure with:

\user{./configure --mandir=/usr/share/man --bindir=/usr/bioprogs/bin}

That's really all you need to know, since \software{infernal} installs
so few files. But just so you know; GNU configure is very flexible,
and has shortcuts that accomodates several standard conventions for
where programs get installed. One common strategy is to install all
files under one directory, like the default \prog{/usr/local}. To
change this prefix to something else, say \prog{/usr/mylocal/}
(so that programs go in \prog{/usr/mylocal/bin} and man pages in
\prog{/usr/mylocal/man/man1}, you can use the \prog{--prefix}
option:

\user{./configure --prefix=/usr/mylocal}

Another common strategy (especially in multiplatform environments) is
to put programs in an architecture-specific directory like
\prog{/usr/share/Linux/bin} while keeping man pages in a shared,
architecture-independent directory like \prog{/usr/share/man/man1}.
GNU configure uses \prog{--exec-prefix} to set the path to
architecture dependent files; normally it defaults to being the same
as \prog{--prefix}. You could change this, for example, by:

\user{./configure --prefix=/usr/share --exec-prefix=/usr/share/Linux/}\\

In summary, a complete list of the \prog{./configure} installation
options that affect \software{infernal}:

\begin{tabular}{lll}
Option                       &   Meaning                       & Default\\ \hline
\prog{--prefix=PREFIX}       & architecture independent files  & \prog{/usr/local/} \\
\prog{--exec-prefix=EPREFIX} & architecture dependent files    & EPREFIX\\
\prog{--bindir=DIR}          & programs                        & PREFIX/bin/\\
\prog{--mandir=DIR}          & man pages                       & PREFIX/man/\\
\end{tabular}


\subsubsection{setting compiler and compiler flags}

By default, \prog{configure} searches first for the GNU C compiler
\prog{gcc}, and if that is not found, for a compiler called \prog{cc}. 
This can be overridden by specifying your compiler with the \prog{CC}
environment variable.

By default, the compiler's optimization flags are set to
\prog{-g -O2} for \prog{gcc}, or \prog{-g} for other compilers.
This can be overridden by specifying optimization flags with the
\prog{CFLAGS} environment variable. 

For example, to use an Intel C compiler in
\prog{/usr/intel/ia32/bin/icc} with 
optimization flags \prog{-O3 -ipo}, you would do:

\user{env CC=/usr/intel/ia32/bin/icc CFLAGS="-O3 -ipo" ./configure}

which is the one-line shorthand for:

\user{setenv CC     /usr/intel/ia32/bin/icc}\\
\user{setenv CFLAGS "-O3 -ipo"}\\
\user{./configure}

If you are using a non-GNU compiler, you will almost certainly want to
set \prog{CFLAGS} to some sensible optimization flags for your
platform and compiler. The \prog{-g} default generated unoptimized
code. At a minimum, turn on your compiler's default optimizations with
\prog{CFLAGS=-O}.

\subsubsection{turning on Message Passing Interface (MPI) support}

\software{infernal} includes four programs \prog{cmsearch}, \prog{cmcalibrate},
\prog{cmalign} and \prog{cmscore} that optionally use MPI
parallelization by invoking the \prog{--mpi} option. 
To enable the option to use MPI in these four executables, add 
\prog{--enable-mpi} to the configuration command:

\user{./configure --enable-mpi}

To run a program in MPI mode, you must run them 
in an MPI environment with \prog{mpirun} or \prog{mpiexec}, with 
the \prog{--mpi} option enabled.  For example, 
in our LAM environment:

\user{mpirun C cmsearch --mpi query.cm target.fa}

Other environments besides LAM MPI should work also, but may
require different command syntax.

\subsubsection{No longer supported: rigorous filters}

Previous versions of \software{infernal} included programs by Zasha
Weinberg that implement rigorous filtering.  The 1.0 release does not
include these programs. If you'd like to use them you can either
download Zasha's own implementation in \textsc{RaveNnA} from
\url{http://bliss.biology.yale.edu/~zasha/ravenna/},
download an older 0.x version of \software{infernal}, or try to modify
this version to work with rigorous filters (the code is still included in
\texttt{rigfilters/}).

\subsection{Example configuration}

The Intel GNU/Linux version installed at Janelia Farm is configured as
follows:

{\scriptuser{env CFLAGS="-O3" ./configure --enable-mpi --enable-lfs --prefix=/usr/local/infernal-1}}





